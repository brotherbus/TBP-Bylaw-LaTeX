\documentclass[12pt]{constitution}
\usepackage{mathpazo}

\begin{document}
\title{Chapter Bylaws\\The Tau Beta Pi Association\\Beta of the District of Columbia}
\author{Revised on \today}
\date{}
\maketitle
%\setcounter{tocdepth}{0}

\bylaw{General}
\section{}
These bylaws shall govern the proceedings of this Chapter in all matters not specifically provided for in the Constitution and Bylaws, and in the Convention Acts of the Tau Beta Pi Association. 
\section{}
This Chapter shall conform to such rules and regulations of the School of Engineering and Architecture and this University as may apply to Honor societies. 
\section{}
In the event of a conflict between the Constitution and Bylaws of the Tau Beta Pi Association, and the rules and regulations of the School of Engineering and Architecture: and this University, such action shall be taken as deemed advisable by all parties concerned. 

\bylaw{Government}
\section{} % Removed Cataloguer and Architecture from name
The officers of the Chapter shall be a President, Vice President, Recording Secretary, Corresponding Secretary, and Treasurer, who shall be active members, and an Advisory Board, as specified in C-VI of the national Constitution. The members of the Advisory Board shall be from the faculty of the School of Engineering.
\section{}
Except for the members of the Advisory Board, the officers of this Chapter shall hold office for one year.
\section{}
The Chapter President shall be the delegate to the national convention. The Chapter officers, shall be alternates in the order listed in Section 1 above.
\section{}
The duties of the officers shall be as outlined in 3-111 of the national Bylaws, and in the bylaws of this Chapter. 
\section{}
The following extra duties shall be required of the officers:
\begin{enumerate}
	\item The Chapter President shall be a member, ex-officio, of all committees. 
	\item The President shall see that each officer and committee chairman is provided with a written list of all specific duties, for which each is responsible. These lists shall be discussed at the first meeting in the fall. A copy of each list shall be placed in the Recording Secretary's notebook and also in the President's book. The President shall also present a letter of recommendation concerning Chapter activity to the membership at the beginning of each semester.
	\item The President shall notify the Advisory Board of the time set for election of new members at least one week in advance.
	\item The Corresponding Secretary shall keep an accurate, up-to-date list of the names
	and addresses of all active members of the Chapter. He shall post notice of all
	meetings at least five days in advance. He shall be the keeper of all dies and stationery
	of the Chapter (see also VIII-1 of these bylaws). He shall also send a copy of all
	resolutions, minutes of each meeting, and correspondence from the President to each active member and to each Advisory Board member.
	\item The Recording Secretary shall keep an up-to-date record of all minutes and correspondence between the President, Advisory Board and membership. He shall also see that each new member is provided with an up-to-date copy of these bylaws for his permanent keeping.
\end{enumerate}
\section{}
The Treasurer may be a faculty member from the School of Engineering and Architecture to hold office for three years. He may be re-elected. He may also be an Advisory Board member.
\section{}
The Project Master shall be the Vice President.
\section{}
No officer of the Chapter, which the exception of the Treasurer, shall succeed himself in the same post.
\section{}
The Advisory Board members shall hold office for four years. Election of one member of the advisory Board will be held each year. 

\bylaw{Meetings}
\section{} 
The following regular meetings shall be scheduled and held once each year: organization meeting, election of officers, and planning meeting. The organization meeting shall be held within two weeks after the beginning of the fall semester. The planning meeting shall be held soon after the election of officers. 
\section{}
The following regular meetings shall be scheduled and held once in the fall and once in the spring: election of candidates, introduction of electees, formal initiation banquet, with the exception that the banquet be held only once after the spring initiation.
\section{}
Meetings shall be held at such times that a majority of the active membership
in be present. 
\section{} The first meeting of each term shall be held within two weeks after the start of that  term.
\section{}
A complete calendar of the regular functions of the Chapter for the term shall be presented to the members by the Chairman of the Program committee no later than the second meeting of the regular term. 
\section{}
Special meetings may be called at any time by the President, or by any member of the Advisory Board, or upon written request to the President signed by three active members.
\section{}
All active members and the Advisory Board shall be notified of all the meetings all in advance by the Corresponding Secretary (see 11-5d of the bylaws). 
\section{}
Robert's Rules of Order shall be the parliamentary guide of this Chapter for all meetings not provided for in these bylaws, and the Constitution and Bylaws of the Tau Beta Pi
association. 
\section{}
Business meetings of the Chapter shall last no longer than two hours, unless extended by a three-fourths vote of all those present.
\section{}
At all meetings, except for the open meetings, the order of business shall be as follows a) Roll call, b) Minutes of the previous meeting, c) Report of officers, d) Reports of committees, e) Unfinished business, f) New business, g) Adjournment.
\section {}
A quorum shall be as set forth in B-111, 16. 

\bylaw{Election of Officers}
\section{}
Officers shall be elected in the spring, no later than April 30, and serve one year. All active members in good standing shall be eligible to hold office.
\section{}
The new officers shall take office at the meeting following the election meeting.
\section{}
Nominations for officers shall be made by any active member and seconded by another active member.
\section{}
The election of officers shall be by secret ballot. Three-fourths of the total active membership shall constitute a quorum for election of officers, and a majority of quorum shall be required for election. If no nominee receives a majority on the first ballot, a second ballot shall be held between the two leading candidates. At this election the new member of the Advisory Board shall also be elected.
\section{}
If any office shall become vacant between the regular elections, a special election will be held as soon as possible. % I had to add in this part, it was cut off

\bylaw{Committees}
\section{}
Subject to the provisions in B-111, 2, the President shall appoint the chairmen and members of the following committees: Membership, Initiation, Program, Social Activities, Alumni Relations, Publicity, and such other committees as the Chapter shall desire and establish. Appointments shall be made at the first meeting after the election of officers, and at the first meeting after the fall initiation of new members. 
\section{}
As early as possible after committee appointments the President shall provide each committee chairman with a list of his specific duties and responsibilities (See II-5b of these bylaws).
\section{}
The special duties of the above named committees shall be as follows:
\begin{enumerate}
	\item Program Committee. It shall be the duty of the Program Committee to meet once each term with the Chapter Executive council to determine a complete calendar of the regular functions of the Chapter (See B-III-5 of these bylaws).
	\item Membership Committee. It shall be the duty of the Membership Committee to render available to the President all information required by B-VII, 3-9 of the bylaws as soon as it is available.
	\item Initiation Committee. It shall be the duty of the Initiation Committee to obtain a location for initiation, to obtain the necessary equipment and material, to ascertain those persons who will participate in the initiation, and to brief the pledges as to the procedure and hold one practice initiation within the appropriate limits of secrecy. 
	\item Publicity Committee. It shall be the duty of the Publicity Committee to see to it that all Chapter projects, functions, and other related activities, whether social or technical shall be publicized in a fitting manner, both on campus and in public press when possible.
	\item Social Activities Committee. It shall be the duty of the Social Activities Committee to program, make appropriate arrangements, and publicize all social activities, including ticket sales where requiered. This shall inclide the spring banquet, lectures and public meetings.
	\item Alumni Relations Committee. It shall be the duty of the Alumni Relations Committee to maintain an up-to-date address list of all Tau Beta Pi alumni, to inform them of all chapter projects and functions, names of new members, and all elections. In addition, it will solicit suggestions and ideas on chapter activities.
	\item Tour Committee. It shall be the duty of the Tour Committee to have material available to the membership which enumerates the faculty, laboratories, research activity, and other associated information of each department of the School of Engineering.
\end{enumerate}

\bylaw{Election}

\section{}
The election of new members shall be held in the fall and spring terms as soon as possible after the grades for the past term become available.
\section{}
All the provisions of C-VIII and B-VI shall be strictly followed. Only students who have met all the other requirements and have a 3.00 accumulative average or above at the time of election will be eligible for membership in this Chapter.
\section{}
% Update this section to delete Arch and add
% Biomedical Engineering
Students in good standing in the following departments, and only these departments, shall be eligible for membership in this Chapter: Architecture, Civil Engineering, Electrical Engineering, and Mechanical Engineering.
\section{}
In deciding the time that certain students shall be eligible for membership, the academic year shall be divided into two equal terms, as defined in the University catalog.
\section{}
Only active undergraduate members are eligible to vote on new members.
\section{}
In the fall, the top fifth of the Senior class shall be eligible for membership. The definition of Senior shall be in accordance with C-VIII, 2s.
\section{}
The top eighth of the Junior class shall be eligible for membership. The order of consideration shall begin with that candidate having the highest grade-point average, and
continuing with the next highest and so on until the quota is filled. The definition of Junior shall be in accordance with C-VIII, 2r.
\section{} % Ask about this, should this matter?
           % Consult current TBP bylaws
The president shall direct the Membership Committee. All members shall keep the
election results in absolute confidence so that no elected student shall learn of his election
except by means of the official letter; likewise he shall not be informed of the details of the vote, especially concerning the personal matters discussed at the time of the voting.
\section{}
Each electee shall be required to notify the President of his acceptance.
\section{}
Each electee who declines for financial reasons shall be interviewed by the Advisory Board, as required by C-VIII, 10.
\section{}
Election of suitable alumni and eminent engineer and architectural members shall be encouraged by this Chapter.

\bylaw{Duties of Pledges}
\section{}
The period of pledging shall begin with an informal pledge installation and continue for a time not to exceed six weeks and will culminate with the formal initiation.
\section{}
In addition to the duties outlined by B-VIII, 1-6, the pledges will be liable to the following duties:
\begin{enumerate}
	\item Each pledge will have to take a quiz on the Constitution and Bylaws and Eligibility Code of the Tau Beta Pi Association, and a brief history of the Association. Initiation will be contingent on his satisfactory completion of this examination.
	\item Each pledge will be liable for service to the Chapter related to Chapter projects
	and activities as determined by the Project Master. Initiation will be contingent on satisfactory participation. No inordinate amount of work will be assigned to either one particular pledge or the pledge class as a whole. The Advisory Board will be the judge as to what constitutes an inordinate amount of work.
\end{enumerate}
\section{}
The Project Master will meet with the pledge c;ass as a whole at least two times during the pledging period, exclusive of the informal installation and the practice initiation.

\bylaw{Records}
\section{}
Records shall be kept up-to-date and accurate (See B-III -1c). The complete records shall be turned over to the new officers before they assume office.
\section{}
All records shall be open for inspection to any member of Tau Beta Pi in good standing.

\bylaw{Finances}
\section{}
The expenses of the Chapter shall be defrayed by the initiation fee, and by such dues and pro rata assessments as may be voted by the Chapter. A majority vote of the total undergraduate membership shall be required to change any fees or dues or levy any assessment. Within one week the Corresponding Secretary shall inform the Secretary-Treasurer of the Association concerning the changes in the amounts of these dues and assessments.
\section{}
Expenditures other than those for less than \$5.00 (which may be made from petty cash) shall be made by check, signed by the Treasurer. Petty cash vouchers must be retained on file for at least three years. Receipts must be written for all money received and must be retained on file for at least three pears.
\section{}
This Chapter shall use the official bookkeeping system of the Association. 
\section{} % Update to current price
The initiation fee for all initiates shall be \$50.00, payable in advance. This shall include all National fees, and the cost of one initiation banquet.
\section{}
There shall be at all times a balance of at least \$9.00 in the treasury. A sum of no more than \$20.00 may be kept in petty cash.
\section{}
Shortly after the spring election of officers, a committee composed of incoming and outgoing Presidents and Treasurers shall prepare an operating budget for the coining year. This budget shall be submitted to the Chapter for approval by a majority vote at the first regular meeting in the fall. Any additional expenses not provided for by the budget must be approved by the Chapter, with these exceptions: the Treasurer shall be
authorized to advance a reasonable sum to the Convention delegate and to settle assessments by the Tau Beta Pi Association.

\bylaw{Discipline}
\section{}
The methods of discipline of a member shall be either suspension or expulsion.
\section{}
Failure to attend 2/3 of the meetings held in the academic year automatically means suspension subject to approval of the Advisory Board. Suspension shall entail the following: (1) Return of badge and certificate (2) Loss of voting privileges (3) Provision by the Chapter of opportunity for the individual to demonstrate his desire to be reinstated. All other provisions stated in the Constititution about suspension shall be followed. Only in cases of sickness or being out of town or any other major difficulty,
will the President excuse a member from a meeting. A written notice in advance will be necessary. Failure to give this notice will constitute an unexcused absence and he will be required automatically to pay a \$2.00 fine within seven days of the unexcused absence. If he fails to pay this fine in the allotted time, his name will be brought before the membership at the next meeting and he will be required to show cause why he should not be considered for suspension or expulsion.
\section{} % Make pronouns gender neautral
           % Not sure if matters too much
In addition to .Section 2, for continued absence from the meetings of the Chapter, for moral delinquency, for inexcusable failure to meet his financial obligations to the Chapter, or for other just cause, after a fair trial and upon the recommendations of a majority of the Chapter and the Advisory Board, his name shall be submitted to
Executive Council of the Association for suspension and expulsion (See C-XIV, 3-4).

\bylaw{Special Projects}
\section{}
This section shall enumerate those special projects which are adopted by the membership by a 2/3 majority vote, subject to approval by the Advisory Board.
\section{}
The undergraduate membership shall avail themselves to act as tour guides of the Pangborn Building and associated laboratories at the request of the Dean of the School of Engineering and Architecture.
\section{} % Don't see the point, might remove or just leave be
The Chapter shall present annually the Outstanding Freshman Engineer Awaxd; the award to be given to that student who attained the highest grade point average of the previous freshman class and exhibited exemplary character. The recipient shall be a full time student at the time of presentation of the award.

\bylaw{Amendments}
\section{}
These bylaws may be amended by a 3/4 vote of the total active membership of the Chapter, subject to the approval of the Advisory Board (see C-VII, 4). Proposed amendments must be submitted to the Chapter in a scheduled meeting at least one week
before the voting. Absentee ballots may be used if necessary.

\end{document}